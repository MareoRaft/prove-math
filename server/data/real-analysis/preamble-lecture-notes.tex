\documentclass[12pt]{article}%

%%Preamble%%

%--------------------------------------------------------
%% File Packages
\usepackage[T1]{fontenc}
%\usepackage{fontspec}
%\usepackage{lmodern}
\usepackage{nameref}
%\usepackage[pdftex,breaklinks]{hyperref}
\usepackage[breaklinks]{hyperref}
\usepackage{cleveref}

%--------------------------------------------------------
%% Misc Packages
%\usepackage{fancyhdr} % Used below
\usepackage{extramarks}

%--------------------------------------------------------
%% Formatting Packages
\usepackage{outlines}
\usepackage{verbatim}%
\usepackage{titling}%

%--------------------------------------------------------
%% Math Packages
\usepackage{amsmath}
\usepackage{amsthm}
\usepackage{amsfonts}
\usepackage{amssymb} % more symbols and fonts
\usepackage{thmtools}

%--------------------------------------------------------
%% Drawing Packages
\usepackage{tikz}
\usepackage[plain]{algorithm}
\usepackage{algpseudocode}

%--------------------------------------------------------
%% Enumerate Packages
% Find out the difference between the two
\usepackage{enumerate} %% To customize enumeration. 
\usepackage[shortlabels, inline]{enumitem} %% From homework Preamble - check it for comment

%--------------------------------------------------------
%%%%%%% Pagestyle stuff %%%%%%%%%%%%%%%%%%%
 \usepackage{fancyhdr}                   %%
   %\pagestyle{fancy}                     %%
   %\fancyhf{} %delete the current section for header and footer
 \usepackage[paperheight=11in,%          %%
             paperwidth=8.5in,%          %%
             outer=1.2in,%               %%
             inner=1.2in,%               %%
             bottom=.7in,%               %%
             top=.7in,%                  %%
             includeheadfoot]{geometry}  %%
   \addtolength{\headwidth}{.75in}       %%
   \fancyhead[RO,LE]{\thepage}           %%
   \fancyhead[RE,LO]{\sectionname}       %%
   \setlength{\headheight}{15.8pt}       %%
   \raggedbottom                         %%
%%%%%%% End Pagestyle stuff %%%%%%%%%%%%%%%

%%% These three lines load and resize a caligraphic font %%%%%%%%%
%%% which I use whenever I want lowercase \mathcal %%%%%%%%%%%%%%%
 \DeclareFontFamily{OT1}{pzc}{}                                 %%
 \DeclareFontShape{OT1}{pzc}{m}{it}{<-> s * [1.100] pzcmi7t}{}  %%
 \DeclareMathAlphabet{\mathpzc}{OT1}{pzc}{m}{it}                %%
                                                                %%
%%% and this is manfnt; used to produce the warning symbol %%%%%%%
 \DeclareFontFamily{U}{manual}{}                                %%
 \DeclareFontShape{U}{manual}{m}{n}{ <->  manfnt }{}            %%
 \newcommand{\manfntsymbol}[1]{%                                %%
    {\fontencoding{U}\fontfamily{manual}\selectfont\symbol{#1}}}%%
%%%%%%%%%%%%%%%%%%%%%%%%%%%%%%%%%%%%%%%%%%%%%%%%%%%%%%%%%%%%%%%%%%

%-------------------------------------------------------- 
%% Alphabet Mappings

\newcommand{\cA}{\mathcal{A}}
\newcommand{\cB}{\mathcal{B}}
\newcommand{\cC}{\mathcal{C}}
\newcommand{\cD}{\mathcal{D}}
\newcommand{\cE}{\mathcal{E}}
\newcommand{\cF}{\mathcal{F}}
\newcommand{\cG}{\mathcal{G}}
\newcommand{\cH}{\mathcal{H}} 
\newcommand{\cI}{\mathcal{I}}
\newcommand{\cJ}{\mathcal{J}}
\newcommand{\cK}{\mathcal{K}}
\newcommand{\cL}{\mathcal{L}}
\newcommand{\cM}{\mathcal{M}}
\newcommand{\cN}{\mathcal{N}}
\newcommand{\cO}{\mathcal{O}}
\newcommand{\cP}{\mathcal{P}}
\newcommand{\cQ}{\mathcal{Q}}
\newcommand{\cR}{\mathcal{R}}
\newcommand{\cS}{\mathcal{S}}
\newcommand{\cT}{\mathcal{T}}
\newcommand{\cU}{\mathcal{U}} 
\newcommand{\cV}{\mathcal{V}}
\newcommand{\cW}{\mathcal{W}}
\newcommand{\cX}{\mathcal{X}}
\newcommand{\cY}{\mathcal{Y}}
\newcommand{\cZ}{\mathcal{Z}}

\newcommand{\A}{\mathbb{A}}
\newcommand{\B}{$\mathbb{B}$}
\newcommand{\C}{\mathbb{C}}
\newcommand{\D}{$\mathbb{D}$}
%\newcommand{\E}{$\mathbb{E}$}
\newcommand{\F}{$\mathbb{F}$}
%\newcommand{\G}{$\mathbb{G}$}
% \renewcommand{\H}{$\mathbb{H}$} % old \H{x} is an x with a weird umlaut in text mode
\newcommand{\I}{\mathbb{I}}
\newcommand{\J}{$\mathbb{J}$}
\newcommand{\K}{$\mathbb{K}$}
% \renewcommand{\L}{$\mathbb{L}$}
\newcommand{\M}{$\mathbb{M}$}
\newcommand{\N}{\mathbb{N}}
% \renewcommand{\O}{$\mathbb{O}$}
\renewcommand{\P}{\mathbb{P}}
\newcommand{\Q}{\mathbb{Q}}
\newcommand{\R}{\mathbb{R}}
% \renewcommand{\S}{$\mathbb{S}$}
\newcommand{\T}{$\mathbb{T}$}
%\newcommand{\U}{$\mathbb{U}$}
\newcommand{\V}{$\mathbb{V}$}
\newcommand{\W}{$\mathbb{W}$}
\newcommand{\X}{$\mathbb{X}$}
\newcommand{\Y}{$\mathbb{Y}$}
\newcommand{\Z}{\mathbb{Z}}

\newcommand{\sA}{\mathsf{A}}
\newcommand{\sB}{\mathsf{B}}
\newcommand{\sC}{\mathsf{C}}
\newcommand{\sD}{\mathsf{D}}
\newcommand{\sE}{\mathsf{E}}
\newcommand{\sF}{\mathsf{F}}
\newcommand{\sG}{\mathsf{G}}
\newcommand{\sH}{\mathsf{H}} 
\newcommand{\sI}{\mathsf{I}}
\newcommand{\sJ}{\mathsf{J}}
\newcommand{\sK}{\mathsf{K}}
\newcommand{\sL}{\mathsf{L}}
\newcommand{\sM}{\mathsf{M}}
\newcommand{\sN}{\mathsf{N}}
\newcommand{\sO}{\mathsf{O}}
\newcommand{\sP}{\mathsf{P}}
\newcommand{\sQ}{\mathsf{Q}}
\newcommand{\sR}{\mathsf{R}}
\newcommand{\sS}{\mathsf{S}}
\newcommand{\sT}{\mathsf{T}}
\newcommand{\sU}{\mathsf{U}} 
\newcommand{\sV}{\mathsf{V}}
\newcommand{\sW}{\mathsf{W}}
\newcommand{\sX}{\mathsf{X}}
\newcommand{\sY}{\mathsf{Y}}
\newcommand{\sZ}{\mathsf{Z}}
\newcommand{\cl}{\mathfrak{cl}}
\newcommand{\g}{\mathfrak{g}}
\newcommand{\h}{\mathfrak{h}}
% \newcommand{\l}{\mathfrak{l}}
\newcommand{\m}{\mathfrak{m}}
\newcommand{\n}{\mathfrak{n}}
\newcommand{\p}{\mathfrak{p}}
\newcommand{\q}{\mathfrak{q}}
% \renewcommand{\r}{\mathfrak{r}} % old \r{x} is a little circle over x in text mode

%--------------------------------------------------------
%% Theorem Tweaks - Gotten from Lecture Notes by Christopher J. Schommer-Pries

\newcounter{homework}
 \renewcommand{\theequation}{\thesection.\arabic{equation}}         %%
 \renewcommand{\thehomework}{\thesection.\arabic{homework}} 
 \makeatletter                                                      %%
    \@addtoreset{equation}{section} % Make the equation counter reset each section
    \@addtoreset{homework}{section} % Make the homework counter reset each section

 \newenvironment{warning}[1][]{%                                    %%
    \begin{trivlist} \item[] \noindent%                             %%
    \begingroup\hangindent=2pc\hangafter=-2                         %%
    \clubpenalty=10000%                                             %%
    \hbox to0pt{\hskip-\hangindent\manfntsymbol{127}\hfill}\ignorespaces%
    \refstepcounter{equation}\textbf{Warning~\theequation}%         %%
    \@ifnotempty{#1}{\the\thm@notefont \ (#1)}\textbf{.}            %%
    \let\p@@r=\par \def\p@r{\p@@r \hangindent=0pc} \let\par=\p@r}%  %%
    {\hspace*{\fill}$\lrcorner$\endgraf\endgroup\end{trivlist}}     %%

\newenvironment{exercise}{
	\refstepcounter{homework}
	\begin{trivlist}  						                        %%
    \item{\bf Exercise~\thehomework.}}{\end{trivlist}}     

 \newenvironment{solution}{\begin{trivlist}%                        %%
    \item{\it Solution:}}{\end{trivlist}}                           %%

 \def\newprooflikeenvironment#1#2#3#4{%                             %%
      \newenvironment{#1}[1][]{%                                    %%
          \refstepcounter{equation}                                 %%
          \begin{proof}[{\rm\csname#4\endcsname{#2~\theequation}%   %%
          \@ifnotempty{##1}{\the\thm@notefont \ (##1)}\csname#4\endcsname{.}}]%%
          \def\qedsymbol{#3}}%                                      %%
         {\end{proof}}}                                             %%
 \makeatother                                                       %%
                                                                    %%

 \newprooflikeenvironment{definition}{Definition}{$\diamond$}{textbf}%
 %\newprooflikeenvironment{example}{Example}{$\diamond$}{textbf}     %%
 \newprooflikeenvironment{remark}{Remark}{$\diamond$}{textbf}       %%
 \newprooflikeenvironment{digression}{Digression}{$\diamond$}{textbf}%
 \newprooflikeenvironment{aside}{Aside}{$\diamond$}{textbf}%
 \newprooflikeenvironment{claim}{Claim}{$\diamond$}{textbf}%

%--------------------------------------------------------
%% Theorem Like Environments

\theoremstyle{plain}               %%%%% Theorem-like commands
\newtheorem{theorem}[equation]{Theorem}
\newtheorem*{theorem*}{Theorem}
\newtheorem{corollary}[equation]{Corollary}
\newtheorem{lemma}[equation]{Lemma}
\newtheorem{proposition}[equation]{Proposition}
\newtheorem{conjecture}[equation]{Conjecture}
\newtheorem{criterion}[equation]{Criterion}
\newtheorem{axiom}[equation]{Axiom}
\newtheorem{principle}[equation]{Principle}


  %\newtheorem{solution}[equation]{Solution}


\theoremstyle{remark} %%%%%% Remark-Like Commands
%\newtheorem{remark}{Remark} % Defined above section
\newtheorem{note}{Note}
\newtheorem{notation}{Notation}
%\newtheorem{claim}{Claim}
\newtheorem{summary}{Summary}
\newtheorem*{acknowledgement}{Acknowledgement}
\newtheorem{case}{Case}
\newtheorem{conclusion}{Conclusion}
 
 % from Lecture Notes by Christopher J. Schommer-Pries
 \newtheorem{question}[equation]{Question}
 \newtheorem{idea}[equation]{Idea}

% See above section for definition like theorem environments

\theoremstyle{definition} 
%\newtheorem{definition}{Definition}
%\newtheorem{define}{Definition}
%\newtheorem*{define*}{Definition}
%\newtheorem{condition}{Condition}
\newtheorem{problem}{Problem}
\newtheorem*{problem*}{Problem}
%\newtheorem{exercise}{Exercise}


\newtheoremstyle{break}%
{}{}%
{}{}%
{\bfseries}{}% % Note that final punctuation is omitted.
{\newline}{}

% Ntheorem solution
%\theoremstyle{nonumberbreak} 
%\theorembodyfont{\normalfont}

% Amsthm solution
\theoremstyle{break}
\newtheorem*{assign}{Assignment}
\newtheorem{example}[equation]{Example}


%--------------------------------------------------------%
%% Inserting Code

% http://stackoverflow.com/questions/3175105/how-to-insert-code-into-a-latex-doc
% From the above
\usepackage{listings}
\usepackage{color}

\definecolor{dkgreen}{rgb}{0,0.6,0}
\definecolor{gray}{rgb}{0.5,0.5,0.5}
\definecolor{mauve}{rgb}{0.58,0,0.82}

\lstset{frame=tb,
  language=Java,
  aboveskip=3mm,
  belowskip=3mm,
  showstringspaces=false,
  columns=flexible,
  basicstyle={\small\ttfamily},
  numbers=none,
  numberstyle=\tiny\color{gray},
  keywordstyle=\color{blue},
  commentstyle=\color{dkgreen},
  stringstyle=\color{mauve},
  breaklines=true,
  breakatwhitespace=true,
  tabsize=3
}

%-------------------------------------------------------- 
%% Various Helper Commands
% For partial derivatives
\newcommand{\pderiv}[2]{\frac{\partial #1}{\partial #2}}

\newcommand{\attention}[1]{[[\ensuremath{\bigstar\bigstar\bigstar} #1]]}  %%% Three Eye-Catching Stars
%-------------------------------------------------------- 
%% Misc - Other Math Operators

% From ChromaticArchive - CourseNotesPreamble
% Lecture Notes by Christopher J. Schommer-Pries
\setlength{\marginparwidth}{.8in}
\definecolor{MyBlue}{rgb}{.1,0.7,1.3}
\newcommand{\notetoself}[1]{\marginpar{\tiny\color{MyBlue}{ #1}}}

% Adding Divides symbols
% http://tex.stackexchange.com/questions/117032/divides-not-divides-and-cardinalities
\DeclareFontFamily{U}{matha}{\hyphenchar\font45}
\DeclareFontShape{U}{matha}{m}{n}{
      <5> <6> <7> <8> <9> <10> gen * matha
      <10.95> matha10 <12> <14.4> <17.28> <20.74> <24.88> matha12
      }{}
\DeclareSymbolFont{matha}{U}{matha}{m}{n}
\DeclareMathSymbol{\notdivides}{3}{matha}{"1F}
\DeclareMathSymbol{\divides}{3}{matha}{"17}

%\DeclareMathOperator{\sup}{sup}
%\DeclareMathOperator{\inf}{inf}
\DeclareMathOperator{\var}{Var}
\DeclareMathOperator{\Var}{Var}
\DeclareMathOperator{\cov}{cov}
\DeclareMathOperator{\Ex}{E}
\newcommand{\abs}[1]{|#1|}
