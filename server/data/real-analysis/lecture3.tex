


\section{Archimedian Property}
%% Lecture September 10, 2015

\begin{problem}
	\( A,B \subset \R \) bounded above. \( A \not = \emptyset, B \not = \emptyset \).
	\( C = \left\{ a + b| a \in A, b \in B \right\} \). 
	Prove that \( \sup C = \sup A + \sup B \). 
\end{problem}

\begin{proof}
	\( C \not = \emptyset, \sup A = \alpha, \sup B = \beta \) and 
	\( \forall a \in A, \forall b \in B, a + b \le \alpha + \beta \). 
	C is bounded above then \( \sup C = \gamma \) exists. 
	Further 
	\begin{equation}
		\gamma \le \alpha + \beta 
		\label{eq:Prob1}
	\end{equation}
	Take \( \varepsilon > 0 \). 
	\( \exists \tilde{a} \in A, \tilde{b} \in B \) such that 
	\( \alpha - \varepsilon < \tilde{a} \) and \( \beta - \varepsilon < \tilde{b} \). 
	Then
	\[
		\alpha + \beta - 2 \varepsilon < a + b \le \alpha + \beta \rightarrow
		\alpha + \beta < \alpha + \beta + 2\varepsilon
	\]
	Since \( \tilde{a} + \tilde{b} \in C, \tilde{a} + \tilde{b} \le \gamma  \) so 
	\( \alpha + \beta \le \gamma + 2\varepsilon \). 
	Hence, \( \forall \varepsilon > 0 \), \( \alpha + \beta < \gamma + 2\varepsilon \).
	\begin{equation}
		\alpha + \beta \le \gamma
		\label{eq:Prob2}
	\end{equation}
	Then \ref{eq:Prob1} and \ref{eq:Prob2} gives us \( \alpha + \beta = \gamma \).
\end{proof}

\begin{problem}[from last lecture]
	Prove \( x \le y + \varepsilon, \forall \varepsilon > 0 \rightarrow x \le y \).
\end{problem}

\begin{proof}
	Suppose \( y < x \), \( x-y > 0 \), \( \varepsilon = \frac{x-y}{2} \). 
	Then \( y + \frac{x-y}{2} < x \) - Contradiction.
\end{proof}

Read Appendix to Ch 1 in Rudin for Dedekind Cuts, which are quite hard. The other way to prove
completeness of \( \R \) is using limits of cauchy sequences. 

\begin{proposition}
	Prove that the set \( \N \) is not bounded above
	\label{prop:naturalbounded}
\end{proposition}

\begin{proof}
	If \( \N \) was bounded above, then \(\exists \alpha \in \R \) such that 
	\( \alpha = \sup \N \). 
	Take \( \alpha - 1 \), there is \( m \in \N \) such that \( m > \alpha -1 \).
	Thi smeans \( \alpha < m + 1 = \tilde{m} \in \N \) which contradicts that 
	\( \alpha \) is an upper bound. 
\end{proof}

This means that for every real number \( x \in \R \) there is a postive integer \( n \in \N \)
such that \( n > x \). 

\begin{theorem}[Archmedian Property of Real Numbers]
	The following statements hold
	\begin{enumerate}
		\item If \( x \in \R \) and \( y \in \R \) and \( x > 0 \), then \( \exists n \in \N \)
			such that \( nx > y \). 
		\item If \( x,y \in \R \), and \( x < y \), then \( \exists q \in \Q \) such that
			\( x < q < y \). 
	\end{enumerate}
	\label{thm:Archimedian Property of Real Numbers}
\end{theorem}

\begin{proof}[Proof of 1]
	Consider \( A = \left\{ nx | n \in \N \cup \left\{ 0 \right\} \right\} \). 
	Assume that \( nx \le y \), \( \forall n \). 
	This means \( A \) is bounded above and \( A \not = \emptyset \). 
	\( \sup A = \alpha \) exists. \( \alpha - x < \alpha \) since \( x > 0 \). 
	Then there is a \( m \in \N \cup \left\{ 0 \right\} \) such that 
	\( \alpha - x < mx \). So \( \alpha < (m+1)x \in A \). 
	Which contradicts \( \alpha \) being an upperbound. 
\end{proof}

\begin{proof}[Proof of 2]
	\( x < y \) then \( y - x > 0 \). From part 1, \( \exists n \in \N \)
	such that  
	\begin{equation}
		n (y-x) > 1
		\label{eq:proof2-1}
	\end{equation}
	Furthermore, 
	\begin{equation}
		\exists m_1 \in \N \text{ such that } m_1 > nx
		\label{eq:proof2-2}
	\end{equation}
	\begin{equation}
		\exists m_2 \in \N \text{ such that } m_2 > -nx
		\label{eq:proof2-3}
	\end{equation}
	Then \ref{eq:proof2-2} and \ref{eq:proof2-3} gives us \( -m_2 < mx < m_1 \).
	Hence we choose \( m \in \Z \) such that
	\begin{equation}
		m-1 \le nx < m
		\label{eq:proof2-4}
	\end{equation}
	Through an inductive finite check, opptimize form right then -1.
	Then combining \ref{eq:proof2-1} and \ref{eq:proof2-4} we get
	\[
		nx < m < 1 + nx < ny
	\]
	Dividing by n, yield
	\[
		x < \frac{m}{n} < y \qedhere
	\]
\end{proof}

\begin{note}
	Can say \( nx, ny \) larger length than 1. So integer exists.
\end{note} 

\begin{assign}
	Read in Rudin the following theorem, found on page 10, \S 1.2.1
\end{assign}
\begin{theorem}
	\( x \in \R, x > 0, \exists ! y \in \R \text{ with } y > 0 \)
	such that \( y^{n} = x \) where \( n \in \N \).
\end{theorem}

\begin{definition}
	A {\em finite decmial representation}  of a \( q \in \Q \) is 
	\[
		q = \pm  \left( a_0 + \frac{a_1}{10} + \frac{a_2}{10^2} + \dots + \frac{a_n}{10^n} \right)
	\]
	with \( a_0 \in \N \cup \left\{ 0 \right\}, 0 \le a_1 < 10 \). 
\end{definition}
\begin{theorem}
	Let \( x \ge 0 \), \( x \in \R \) then for every \( n \ge 1 \) integer, 
	there exists a finite decimal \( r_n = a_0.a_1 \dots a_n \) such that
	\( r_n \le x < r_n + \frac{1}{10^n} \).
\end{theorem}

\begin{proof}
	Let \( S = PPall nonnegative integers \le x \). 
	\( 0 \in S \rightarrow S \not = \emptyset \). 
	\( X \) is an upper bound. Let \( a_0 = \sup S \) which exists. 
	\( a_0 = [x] \) the largest integer smaller then \( x \), \( a_0 \ge 0 \). 
	Then \( 10 > 10x - 10a_0 \ge 0 \) because \( x - a_0 < 1 \) by def of \( a_0 \). 
	We also get 
	\( 0 \le a_1 = [10x - 10a_0] \le 9 \). And so 
	\[
		a_0 + \frac{a_1}{10} \le x < a_0 + \frac{a_1}{10} + \frac{1}{10}
	\]
	Then we perform induction to finish the proof.
\end{proof}
\begin{aside}
	\( \frac{1}{10^n}, \abs{x-r_n} < \frac{1}{10^n} \).
	There is significance to a similar yet slightly different statement
	\( r_n < x \le r_n + \frac{1}{10^n} \)
	\attention{infinite representation of rational?}
\end{aside}

\( \R \) extended real line is \( \R \cup \left\{ -\infty \right\} \cup \left\{ +\infty
\right\} \rightarrow does not form a field \). It is used mainly as a convenience when down the
road we will want to have limits to \( + \infty \), \( -\infty \) and neighborhoods of \( \infty
\). It is easy to say that all set bounded through this extension.

\begin{notation}
	\( [a,b], [a,b), [a,+\infty), \) and when \( \infty \) is part of the set, \( [a,+\infty] \)
\end{notation}

\begin{definition}
	The set \( \C \) is the set of {\em complex number} and is a field. 
	Let \( z \in \C \), 
	\( z = (a,b), i = (0,1), z = a + ib, \text{ where } a = (a,0), b = (0,b) \).
	\( z_1 = (a_1, b_1), z_2 = (a_2,b_2) \). Then
	\( z_1 + z_2 = (a_1 + a_2, b_1 + b_2) \) and 
	\( z_1 * z_2 = \left( a_1a_2 - b_1b_2, a_1b_2 + a_2b_1  \right) \) 
	because \( \C \) is isomorphic to matrix multiplicaton. See an algebra text.
	*** Rephrase this, it doesnt make sense now.
\end{definition}

\begin{note}
	\( \C \) is not an ordered field
\end{note}
