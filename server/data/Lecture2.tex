

\section{Ordered Sets, Fields, Ordered Fields}
% Lecture 2 on 09/08/15
\subsection{Ordered Sets}
\begin{example}
	\label{example1}
	Construct \( A = \left\{ p \in \Q | p^2 < 2 \right\} \) and \( B = \left\{ p \in Q | p^2 > 2
	\right\} \). We will show that \( A \) contains no largest number; that is 
	\( \forall p \in A\), we can find a \( q \in A \) such that \( p < q \).
	We will also show \( B \) contains no smallest number;
	that is \( \forall p \in B \), we can find a \( q \in B \) such that \( q < p \).

	Take \( q = p - \frac{p^2-2}{p+2} \) which we choose because we want to control the sign of
	q and of the quotient term. Its clear that after simplifying we get
	\begin{equation}
		q = \frac{2p+2}{p+2} \in Q
		\label{eq1}
	\end{equation}
	as the quotient of rational is rational. Now we compute 
	\begin{equation}
		q^2 - 2 = \frac{4p^2+8p+4}{p^2+4p+4} - 2 = \frac{4p^2 + 8p + 4 - 2p^2 - 8p -8}
		{(p+2)^2} = \frac{2p^2 - 4}{(p+2)^2}
		\label{eq2}
	\end{equation}
	\begin{enumerate}[]
		\item If \( p \in A, p^2 - 2 < 0, \) this means \( q^2-2 < 0 \rightarrow q \in A \).
			Further \ref{eq1} implies \( q > p \). 
		\item If \( p \in B, p^2 - 2 > 0 \), this means \( q^2 - 2 > 0 \rightarrow q \in B \).
			Further \ref{eq1} implies \( p < q \).
	\end{enumerate}
\end{example}

\begin{definition}
	Let S be a set. An {\em order}  in \( S \) is a relation, denoted by ``\( < \)'' that
	satisfies
	\begin{enumerate}[]
		\item If \( x \in S \) and \( y \in S \) then only one of these statements holds true
			\begin{itemize}[]
				\item \( x < y \)
				\item \( y < x \)
				\item \( x = y \)
			\end{itemize}
		\item If \( x < y \) and \( y < z \), then \( x < z \), for \( x,y,z \in S \)
	\end{enumerate}
	\label{order}
\end{definition}

\begin{corollary}
	\begin{enumerate}[]
		\item 	We define ``\( > \)'' as the inverse relation of \( < \).
			That is \( x > y \iff y < x \). 
		\item We define \( \ge \) to represent \( \not < \).
		\item We define \( \le \) to represent \( \not > \).
	\end{enumerate}
	\label{OtherSigns}
\end{corollary}

\begin{example}
	\( Q \) is ordered. We get \( p < q \) if \( p - q < 0 \).
\end{example}

\begin{example}
	\( S, \P(S) = \left\{ A | A \subset S \right\} \). \( A \subset B \) is like \( A < B \).
	This is not ordered because two sets do not have to be subsets of each other.
	\label{Posetex}
\end{example}
\begin{note}
	This last example, \ref{Posetex}, is an example of a partially ordered set, often called a
	poset.
\end{note}

\begin{definition}
	A ordered set \( S \) is a set when an order relation is defined for every element of \(
	S \)
\end{definition}

\begin{definition}
	Let \( S \) be an ordered set and \( E \subset S \)
	\begin{enumerate}[]
		\item \( E \) is bounded above if \( \exists \beta \in S \) such that 
			\( x \le \beta \) for all \( x \in E \). \( \beta \) is called an {\em upper bound}
			of \( E \).
		\item \( E \) is bounded below if \( \exists \gamma \in S \) such that 
			\( \gamma \le x \) for all \( x \in E \). 
			\( \gamma \) is called a {\em lower bound} of \( E \).
	\end{enumerate}
	\label{bounddef}
\end{definition}

\begin{note}
	\( \C \) is a commonly used set (of numbers) that is not ordered
\end{note}

\begin{definition}
	Let \( S \) be an ordered set and \( E \subset S \) bounded above. 
	Then \( \alpha \in S \) is called the {\em least upper bound} of \( E \) if
	\begin{enumerate}[a.]
		\item \( \alpha \) is an upper bound of \( E \)
		\item If \( \gamma < \alpha \) then \( \gamma \) is not an upper bound of \( E \).
	\end{enumerate}
	The least upper bound is denoted by \( \sup{E} \) and is called the {\em supremum}.
	\label{supremum}
\end{definition}

\begin{exercise}
	Prove the supremum is unique.
\end{exercise}

\begin{definition}
	Let \( S \) be an ordered set and \( E \subset S \) bounded below. 
	Then \( \alpha \in S \) is called the {\em greatest lower bound} of \( E \) if
	\begin{enumerate}[a.]
		\item \( \alpha \) is an lower bound of \( E \)
		\item If \( \gamma > \alpha \) then \( \gamma \) is not a lower bound of \( E \).
	\end{enumerate}
	The greatest lower bound is denoted by \( \inf{E} \) and is called the {\em infimum}.
	\label{infimum}
\end{definition}

\begin{example}
	\( E = \left\{ \frac{1}{n} | n \in \N \right\} \). Then \( \sup(E)=1 \in E \) and \(
	\inf(E)=0 \not \in E \), \( E \subset Q \). 
\end{example}

\begin{example}
	\( A = \left\{ p > 0 | p^2 < 2 \right\} \) all positive elements of \( B \) are upper
	bounds, but \( \sup(A) \) does not exist by \ref{example1}
\end{example}

\begin{example}
	\( B = \left\{ p > 0 | p^2 > 2 \right\} \) all negative elements of A are lower bounds, but
	\( \inf(B) \) does not exist by \ref{example1}
\end{example}

\begin{definition}
	We say that an ordered set \( S \) has the {\em least upper bound property} if any bounded
	above subset has the least upper bound in \( S \).
	\label{LUBprop}
\end{definition}

\begin{example}
	\( \Q \) does not have the least upper bound property.
\end{example}

\begin{theorem}
	Let \( S \) be an ordered set (o.s.) with the LUB property and \( B \subset S \), 
	\( B \not = \Q \) and bounded below. Denote by \( L \) the set of all lower bounds of \(
	B \). Then \( \alpha = \sup(L) \) exists and \( \alpha = \inf(B) \).
	\label{Theorem1}
\end{theorem}

\begin{proof}
	Let \( L = \left\{ y \in S | y \le x, \forall x \in B \right\}, L \not = \emptyset \) since
	B is bouded below (it must contatin at least one element). L is bounded above, since \(
	\forall x \in B \) are upper bounds. 
	Hence \( \alpha = \sup L \) exists. 
	We show that \( \alpha = \inf B \). 
	Indeed if \( r < \alpha \) then \( r \) is not an upper bound of \( L \). 
	That is \( \exists B \in L \) such that \( r < B \). This means every \( x \in B \) is 
	\( r < B \le x \). That means \( r \not \in B \).
	Then \( \alpha \in L \) i.e. \( r \le x, \forall x \in B  \), \( r \) is a lower bound. 
	If \( B > \alpha \) then \( B \not \in L \) since \( \alpha \) is an upper bound of \( L
	\) i.e. \( B \) is not a lower bound of \( B \). 
\end{proof} 

\begin{definition}[Fields]
	The set \( F \) equipped with two operations ``+'' addition, ``*'' multiplication is called
	a {\em field} if: \\
	\textbf{Addition Axioms}
	\begin{enumerate}[1)]
		\item If \( x,y \in F \) then \( x+y \in F \) (closure addition)
		\item \( x+y = y+x \) for all \( x,y \in F \) (commnativity addition)
		\item \( (x+y) + z = x + (y+z) \) for all \( x,y,z \in F \) (associative addition)
		\item \( F \) contains an element \( 0 \) such that \( 0 + x = x, \forall x \in F \) 
			(identity)
		\item To every \( x \in F \) it corresponds an element \( -x \in F \) such that 
			\( x + -x = 0 \)
	\end{enumerate}
	\textbf{Multiplicative Axioms}
	\begin{enumerate}[1)]
		\item If \( x,y \in F \), then \( x*y \in F \) (multiplication closure)
		\item \( x *y = y*x, \forall x,y \in F \) (multiplication commutativity)
		\item \( (x*y)*z = x*(y*z), \forall x,y,z \in F \) (multiplication associative)
		\item \( F \) contains an element \( 1 \not = 0 \) such that 
			\( 1 * x = x, \forall x \in F \) (multiplication identity)
		\item \( \forall x \in F, x \not = 0 \) it corresponds an element 
			\( 1/x \in F \) such that \( x * 1/x = 1 \) (mult inverse)
	\end{enumerate} 
	\label{def:Fields}
\end{definition}

\begin{exercise}
	Do problems 1.14, 1.15, 1.16 in Rudin
\end{exercise}
\begin{exercise}
	Show 0*x = 0
\end{exercise}
\begin{proof}
	Find \( x + y = x + z \), then \( y = z \). Then we show
	\[
		y = 0 + y = (x + (-x)) + y = -x + (x + y) = -x + (x + z) = z
	\]
	Now using the above result we get
	\[
	0 * x = (0 + 0)x = 0x + 0x \rightarrow 0x = 0 
	\]
	where we are using the above property with \( z = 0 \). So we get
	\[
		 x+y = x \rightarrow y = 0
	\]
\end{proof}

\subsection{Ordered Fields}
\begin{definition}
	Let \( F \) be a set which is an ordered set under ``<'' and a field under ``+'', ``*'' such
	that 
	\begin{enumerate}
		\item \( x + y < x + z \) if \( x,y,z \in F \) and \( y < z \)
		\item \( xy > 0 \) if \( x,y \in F, x > 0, y > 0 \)
	\end{enumerate}
	\label{def:OrderedFields}
\end{definition}
\begin{exercise}
	Prove if \( x>0 \) then \( (-x) < 0 \)
\end{exercise}
\begin{theorem}
	There exist and ordered field \( \R \) which \( \Q \subset \R \) and has the Least Upper
	Bound Property
	\begin{enumerate}[a)]
		\item Dedekin (Appendix, Ch 1)
		\item Completeness
	\end{enumerate}
	\label{def:R}
\end{theorem}

\textbf{Homework Hints - Prove the following Lemma} 
\begin{enumerate}
	\item If \( a + \varepsilon \le b \) for every \( \varepsilon > 0 \) then \( a \le b \)
	\item If \( a \le b + \varepsilon \) for every \( \varepsilon > 0 \) then \( a \le b \)
	\item Restate inf,sup using limits
\end{enumerate}
