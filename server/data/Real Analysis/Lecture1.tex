

\section{Foundations of Real Numbers}

We will denote from now on the real numbers as \( \R \).

\begin{definition}[Inductive Set]
	A subset, \( A \), of real numbers that satisfies: 
	\begin{enumerate}
		\item \( 1 \in A \)
		\item if \( x \in A \) then \( x + 1 \in A \)
	\end{enumerate}
	is called {\em inductive}.
	\label{inductiveset}
\end{definition}

\begin{definition}
	The smallest inductive set is called the set of {\em positive integers} or {\em natural
	numbers} denoted by \( \N \).
	\label{naturalnumbers} 
\end{definition}

\begin{note}
	Multiple definitons of integers (or in general defintions of any object) should be
	equivalent.
\end{note}

\begin{definition}
	Integers are denoted \( \Z \) and are equal to \( \N \cup -\N \cup \left\{ 0 \right\} \),
	where \( - \N = \left\{ -n | n \in \N \right\} \). 
	\label{Integers}
\end{definition}

\begin{principle}[Induction]
	To prove a statment \( P_n \) for all \( n \in \N \) by the defintion of \( \N \) we need to
	the following:
	\begin{enumerate}
		\item Verify that \( P_1 \) is true.
		\item Assume \( P_{n-1} \) is true, then prove \( P_n \) is also true, for \( n > 1 \).
	\end{enumerate}
	\label{induction}
\end{principle}

\begin{definition}[Properties]
	The following are properties of nautral numbers.
	\begin{enumerate}
		\item \( n = d*c \), then \( d \) {\em divides} \( n \), i.e. is a divisor of \( n \)
		\item \( n \) is {\em prime} if \( n > 1 \) and the only positive divisors are \( n
			\) and \( 1 \). 
	\end{enumerate}
	\label{naturalproperties}
\end{definition}

\begin{theorem}[Factorization in prime factors]
	Every postive integer \( n > 1 \) is either prime or can be represented as a product of
	primes. Furthermore this respresentation is unique up to reordering.
	\label{PrimeFactorization}
\end{theorem}

\begin{proof}
	We use induction to prove the first statement.
	\begin{enumerate}
		\item \( n=2 \), is prime \( \rightarrow \) true.
		\item \( 2, \dots, n-1 \) is true using equivalent form induction to \ref{induction}
			where we assume \( P_k \) for
			\( 2 \le k \le n-1 \).
			\begin{enumerate}[a.]
				\item \( n \) is prime, we are done
				\item Or it has a divisor \( \not = 1 \) and \( n \). That is \( n = c * d \)
					for \( 1 < c < n, 1 < d < n \). Hence \( c \) and \( d \) can be represented
					as a product of primes \( \rightarrow \) so can \( n \).
			\end{enumerate}
	\end{enumerate}
	Now we prove uniquness. Assume that \( n = d_1 d_2 \dots d_s = c_1 c_2 \dots c_t \) for 
	\( d_i, c_j \) are primes. We want to show \( s=t \) and that \( \forall i, \exists ! j, d_i = c_j \), 
	We have 
	\[ 
		d_1d_2 \dots d_s = c_1 c_2 \dots c_t.
	\]
	\( d_1 \) divides the right hand side, hence it divides one of the factors because it is
	prime. Say after rearranging we call the factor that it divides \( c_1 \). Since \( d_1, c_1
	\) are prime, \( d_1 = c_1 \). Divides both sides we get
	\[
		\frac{n}{d_1} = d_2 d_3 \dots d_s = c_2 c_3 \dots c_t
	\]
	Note that \( 2 < \frac{n}{d_1} < n \), therefore from our inductive assumption we conclude
	that \( s = t \) and \( d_i = c_j \) for \( i = 2 \dots s \) and \( j = 2 \dots t \). 
\end{proof}

\begin{theorem}[Infinite Primes - Euclid]
	There are an infinite number of primes.
	\label{Infinite Primes}
\end{theorem}

\begin{proof}
	We prove this using contradiction. Assume to contrary that there are m primes \( 2, 3 \dots
	n_m\) where \( n_m \) is the largest prime. Coniser \( n = 2 * 3 \dots n_m + 1 \). 
	\( n \) cannot be prime because \( n > n_m \). Hence \( n \) is a product of primes, but \(
	n \) cannot be divides by an primes because it leaves a remainder of 1. Contradiction.
\end{proof}

\begin{definition}[Rational Numbers]
	{\em Rational Numbers}  are quotients of integers \( \frac{m}{n}, n \not = 0 \) denoted as
	\( \Q \).
	\label{Rational}
\end{definition}

\begin{corollary}[Properties]
	The following are properties of Rational Numbers.
	\begin{enumerate}[]
		\item \( \frac{a+b}{2} \) is rational if \( a \in Q \) and \( b \in Q \).
		\item between any two rational there are infinitely many rationals. 
	\end{enumerate}
	\label{RationalProperties}
\end{corollary}

\begin{example}
	Show that there are no rational roots of \( p^2 + 2 \).
\end{example}

\begin{solution}
	We solve \( p^2 - 2 = 0 \). Assume by contradiction that there is a rational root and let it
	be represented as \( \frac{m}{n} \in \Q \) such that \( \frac{m}{n}^2 = 2 \). We can choose
	\( m,n \) such that they are coprime (i.e. they have no common factor).  Then we get 
	\( m^2 = 2n^2 \Rightarrow m^2 \) is even \( \Rightarrow m \) is even. Thus it follows that
	\( m = 2p \Rightarrow 4p^2 = 2n^2 \Rightarrow 2p^2 = n^2 \). Hence \( n^2 \) is even \(
	\Rightarrow n \) is even. Therefore we have that \( m \) and \( n \) are both even.
	Contradiction to our assumption that \( m,n \) are coprime.
\end{solution}

\begin{note}
	This shows that \( \sqrt{2} \) is irrational
\end{note}

\begin{exercise}
	Show that if \( m \) is not a perfect square \( m \not = d*d \) then \( \sqrt{m} \) is not
	rational. That is show that \( p^2 = m, p \not \in \Q \).
\end{exercise}

\begin{example}
	Show that \( e \) is not in \( \Q \).
\end{example}

\begin{solution}
	We recall from Calc 2 that 
	\[
		e^x = \sum_{n=1}^{\infty} \frac{x^n}{n!}
	\]
	\attention{which leads to}
	\[
		e = \sum_{n=1}^{\infty} \frac{1}{n!}
	\].
	Thus it suffices to show that \( e^{-1} \not \in \Q \). Recall that
	\[
		e^{-1} = \sum_{n=1}^{\infty} \frac{(-1)^n}{n!}
	\]
	and define \( S_m = \sum_{n=1}^{m} \frac{(-1)^n}{n!} \). It is clear that 
	\[
		0 < e^{-1} - S_{2k-1} < \frac{1}{(2k)!}
	\]
	which leads us to 
	\[
		0 < (2k-1)! (e^{-1}-S_{2k-1}) < \frac{1}{2k} < \frac{1}{2}
	\]

	\( (2k-1)!S_{2k-1} \in \Z \),
	\attention{in fact it is positive because the error term is postive}.
	If \( e^{-1} \) was rational, we can choose \( k \) large enough such that 
	\( (2k-1)!e^{-1} \) is in \( \N \) because \( e^{-1} = \frac{p}{q}, q \divides
	(2k-1)! \) if \( k \) is large enough. We conclude \( (2k-1)!(e^{-1}-S_{2k-1}) \in \Z \) but
	this is not possible because there are no integers between \( 0 \) and 
	\( \frac{1}{2} \).
\end{solution}

\begin{note}
	Transcendental irrational numbers are uncountable which is not the case for algebraic
	irrational numbers
\end{note}
